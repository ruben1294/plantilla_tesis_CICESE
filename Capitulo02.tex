\chapter{Metodolog\'ia}\label{capit:cap2}
\vspace{-2.0325ex}%
\noindent
\rule{\textwidth}{0.5pt}
\vspace{-5.5ex}% 
\newcommand{\pushline}{\Indp}% Indent puede ir o no :p

En esta secc\'on se describen de forma detallada los procedimientos utilizados para la realizaci\'on de la investigaci\'on, con el propósito de que se pueda reproducir. Incluye una descripci\'on de los insumos utilizados por ejemplo: muestras colectadas, mediciones variables en el \'area de estudio o datos disponibles en alguna base de datos de acceso p\'ublico o privado. Esta secci\'on es apropiada para describir diseños experimentales, protocolos de adquisici\'on e instrumentaci\'on empleados. Se escribe en tiempo pasado y no debe ser una lista de materiales ni de pasos a seguir y es conveniente evitar el uso de t\'erminos ambiguos tales como: frecuentemente, regularmente, aproximadamente.
\\

\section{Ecuaci\'on (ejemplo)}\label{secc:ejemploec}

\begin{equation}
\theta(t) = \theta(0) + Zt0 \theta(\tau)dt
\label{eq:ejem}
\end{equation}

\section{Figura (ejemplo)}\label{secc:ejemplofig}

\begin{figure}[h]
        \centering
        \includegraphics[width=100mm]{./Figures/logoCicese2009.pdf}
        \caption{Logo de CICESE} 
				\label{fig:ejemplo1}
\end{figure}

El Centro de Investigación Científica y de Educación Superior de Ensenada, Baja California (CICESE) fue la segunda institución creada por el Consejo Nacional de Ciencia y Tecnología (CONACYT) para descentralizar las actividades científicas y tecnológicas en México”. En la Figura \ref{fig:ejemplo1} se puede ver el logo de CICESE. 

\newpage
%%=====================================================
